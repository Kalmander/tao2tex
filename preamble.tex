% !TeX program = xelatex
\documentclass[11pt]{article}
\usepackage{amsmath,amssymb}
\usepackage{amsthm}



\usepackage{fontspec}%
\newfontfeature{Microtype}{protrusion=default;expansion=default;}%
\setmainfont{texgyrepagella-regular.otf}[%
  Microtype,
  Ligatures = TeX,
  BoldFont = texgyrepagella-bold.otf,
  ItalicFont = texgyrepagella-italic.otf,
  BoldItalicFont = texgyrepagella-bolditalic.otf,
]%
\usepackage{unicode-math}%
\setmathfont{texgyrepagella-math.otf}%

\renewcommand{\baselinestretch}{1.1}
\setcounter{secnumdepth}{0}



\usepackage{enumitem}
\setlist{leftmargin= 1.3em, labelsep=0.5em} % adjust spacing for lists
%%% below are simple theorems that use amsthm only
	\newtheorem{theorem}{Theorem}
	\newtheorem{corollary}[theorem]{Corollary}
	\newtheorem{lemma}[theorem]{Lemma}
	\newtheorem{proposition}[theorem]{Proposition}
	\newtheorem{conjecture}[theorem]{Conjecture}
\theoremstyle{definition}
	\newtheorem{definition}[theorem]{Definition}
	\newtheorem{example}[theorem]{Example}
	\newtheorem{exercise}[theorem]{Exercise}
\theoremstyle{remark}
	\newtheorem{remark}[theorem]{Remark}
	\newtheorem{note}[theorem]{Note}
\usepackage[framemethod=tikz]{mdframed}
% \mdfdefinestyle{tao}{outerlinewidth = 1,roundcorner=2pt,innertopmargin=0}
% 	\newmdtheoremenv[style=tao]{theorem}{Theorem}
% 	\newmdtheoremenv[style=tao]{corollary}[theorem]{Corollary}
% 	\newmdtheoremenv[style=tao]{lemma}[theorem]{Lemma}
% 	\newmdtheoremenv[style=tao]{proposition}[theorem]{Proposition}
% 	\newmdtheoremenv[style=tao]{conjecture}[theorem]{Conjecture}
% \theoremstyle{definition}
% 	\newmdtheoremenv[style=tao]{definition}[theorem]{Definition}
% 	\newmdtheoremenv[style=tao]{example}[theorem]{Example}
% 	\newmdtheoremenv[style=tao]{exercise}[theorem]{Exercise}
% % \theoremstyle{remark}
% 	\newmdtheoremenv[style=tao]{remark}[theorem]{Remark}
% 	\newtheorem{note}[theorem]{Note}
% 	\usepackage[margin=3cm]{geometry}
\usepackage[normalem]{ulem} % needed for strikethroughs
\usepackage{graphicx}
%%%%% If you find emoji in the blogpost (perhaps in the comments), 
%%%%% then you can comment out the next line:
\newcommand{\emoji}[1]{\texttt{#1}} % and instead,
%%%%% use LuaTeX and the emoji package to properly print them:
% \usepackage{emoji}
\usepackage{microtype} % better text formatting
\usepackage{xcolor}
\usepackage[hyphens]{url} % allow linebreaks at hyphens
\usepackage[colorlinks = true,
			citecolor = blue,
			urlcolor = blue,
			linkcolor = blue]{hyperref}
\makeatletter         
\renewcommand\maketitle{
% \noindent {\Large TTT-BLOG-TITLE}\\
% \textcolor{gray}{TTT-TAGLINE}
{\begin{center}
{\Large \bfseries  \@title{}}\\
{\vspace{1ex}}
{\it\footnotesize TTT-METADATA}\\
\end{center}}
% {\tiny TTT-SIGNATURE\\ \hrule  \vspace{4ex}}
{\vspace{1ex}}
}
\makeatother
\title{TTT-TITLE}
