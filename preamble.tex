\documentclass[11pt]{article}
\usepackage{amsmath,amssymb}
\usepackage{amsthm}

\usepackage{enumitem}
\setlist{leftmargin= 1.3em, labelsep=0.5em} % adjust spacing for lists
%%% below are simple theorems that use amsthm only
%	\newtheorem{theorem}{Theorem}
%	\newtheorem{corollary}[theorem]{Corollary}
%	\newtheorem{lemma}[theorem]{Lemma}
%	\newtheorem{proposition}[theorem]{Proposition}
%	\newtheorem{conjecture}[theorem]{Conjecture}
%\theoremstyle{definition}
%	\newtheorem{definition}[theorem]{Definition}
%	\newtheorem{example}[theorem]{Example}
%	\newtheorem{exercise}[theorem]{Exercise}
% \theoremstyle{remark}
%	\newtheorem{remark}[theorem]{Remark}
%	\newtheorem{note}[theorem]{Note}
\usepackage[framemethod=tikz]{mdframed}
\mdfdefinestyle{tao}{outerlinewidth = 1,roundcorner=2pt,innertopmargin=0}
	\newmdtheoremenv[style=tao]{theorem}{Theorem}
	\newmdtheoremenv[style=tao]{corollary}[theorem]{Corollary}
	\newmdtheoremenv[style=tao]{lemma}[theorem]{Lemma}
	\newmdtheoremenv[style=tao]{proposition}[theorem]{Proposition}
	\newmdtheoremenv[style=tao]{conjecture}[theorem]{Conjecture}
\theoremstyle{definition}
	\newmdtheoremenv[style=tao]{definition}[theorem]{Definition}
	\newmdtheoremenv[style=tao]{example}[theorem]{Example}
	\newmdtheoremenv[style=tao]{exercise}[theorem]{Exercise}
% \theoremstyle{remark}
	\newmdtheoremenv[style=tao]{remark}[theorem]{Remark}
	\newtheorem{note}[theorem]{Note}
	\usepackage[margin=3cm]{geometry}
\usepackage[normalem]{ulem} % needed for strikethroughs
\usepackage{graphicx}
%%%%% If you find emoji in the blogpost (perhaps in the comments), 
%%%%% then you can comment out the next line:
\newcommand{\emoji}[1]{\texttt{#1}} % and instead,
%%%%% use LuaTeX and the emoji package to properly print them:
% \usepackage{emoji}
\usepackage{microtype} % better text formatting
\usepackage{xcolor}
\usepackage[hyphens]{url} % allow linebreaks at hyphens
\usepackage[colorlinks = true,
			citecolor = blue,
			urlcolor = blue,
			linkcolor = blue]{hyperref}
\makeatletter         
\renewcommand\maketitle{
\noindent {\Large TTT-BLOG-TITLE}\\
\textcolor{gray}{TTT-TAGLINE}
{\begin{center}
{\Huge \bfseries\sffamily  \@title{}}
\end{center}}
{\noindent\footnotesize TTT-METADATA}\\
{\tiny TTT-SIGNATURE\\ \hrule  \vspace{4ex}}}
\makeatother
\title{TTT-TITLE}